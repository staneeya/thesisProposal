\chapter{Introduction}
\thispagestyle{plain}

\label{Chapter1}

In this chapter, we will explain why this dissertation need to be done in the motivation section, then followed with the thesis statement and contributions that will be made by this research. 

\section{Motivation}
\label{motivation}
In the new world that people are overwhelmed with information from internet such as social media, news articles, video multimedia. We consume these information all day everyday. While we welcome these connections, we also risk ourselves to many patterns of cyber crimes. As we seen from news recently, Yahoo announced in September 2016 that more than one billion accounts’ data were stolen, it is the biggest hacks in the history. According to IBM websites (http://www-03.ibm.com/security/cognitive/),
\\\\
\textit{"The volume of threat data is exceeding the capacity of even the most skilled security professional, and organizations are drowning in a sea of information that continues to grow as rapidly as the threat landscape itself. When organizations see over 200,000 security events every day and don’t have the skills to stay ahead."}\\

These cyber crime happens everyday. How we/human can defend them? We have to build machine to do this job. The first step for training any machine to do their brilliant jobs is gathering useful data, make them structured and retrievable. There are a lot of information about the cyber crime patterns that happened. They are both in the structured format such as National Vulnerability Database, Microsoft Bulletin, and unstructured format such as white paper, news article, hackers forum. We have to have the information schema that can storing these information systematically. Moreover, we need to build system that efficiently extracts the useful information from these data sources.\\
\indent Information extraction researches had been developed many decades. The interested information can be categorized in many aspects, one group that has been continually researched consists of named entity recognition, relation extraction, and event mention detection. Event mention detection or event extraction is the most recent research in this group that been interested. All of event extraction researches are about detection of event mentioned in news or tweets. The events are about human’s life. For example, John married Jane on March 1st, 2017. The extracted information is event:married, participants: John, Jane, time: March 1st, 2017. We also have experience in the human’s life event mention detection research. These human’s life events are different from events that happened in cyber crime.\\
\indent To our knowledge, we have not ever seen any event extraction researches were done in cybersecurity domain before. This dissertation will fulfill the need in gathering cyber crime events information and structured them for useful in building cyberdefense machine in the future. We will focus on designing the cybersecurity event ontologies, and developing a system to extract event information from cyber crime data sources.

\section{Thesis Statement}
\label{statement}
This dissertation will invent the useful and completeness event ontology for cyber security domain and also develop the event extraction system that can extract cyber crime events from various data sources.
\section{Contribution}
\label{contribution}
My Ph.D thesis will contribute to the field of computer science in the area of information extraction and cybersecurity by 
\begin{enumerate}

    \item Invent new ontology for cybersecurity events that
    
    \begin{itemize}
        \item Has clarification definition about event types, roles of each participants, including examples for easy understanding.
        \item Complete for storing every important aspects of cybersecurity events.
        \item Understandable for non-specialist and specialist in cybersecurity domain.
        \item Useful for storing and sharing cybersecurity information.
        \item Reusable for further analysis in predicting of future cybersecurity attacks.    
    \end{itemize}
    \item Develop an event extraction system for cyber security domain that
    \begin{itemize}
        \item Extracting event mentioned in the text.
        \item Detecting participants of event extracted.
        \item Assigning roles for each participants of events.
        \item Classify event to its type or subtype.

    \end{itemize}
\end{enumerate}
 


