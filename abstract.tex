\newpage
\pagestyle{empty}

\begin{center}
\vspace{0.1in}
\large{\bf ABSTRACT} \par  
\bigskip \bigskip
\end{center}

\begin{flushleft}
{\bf Title of Thesis:} 
Event Ontology and Event Extraction System for Cybersecurity\\
Taneeya Satyapanich, Ph.D. Proposal, 2017 \\
\begin{singlespace}
{\bf Thesis directed by:}{\hspace{2.5mm}} \parbox[t]{3in}{Dr. Tim Finin, Professor \\
Department of Computer Science and \\ Electrical Engineering}
\end{singlespace}
\end{flushleft}

In the present, we live by relying on internet such as internet banking, ordering food, socialize. The technology facilitates us but also come with possible cyber crimes such as stolen data, identity theft. With such a lot of transaction done everyday, it also has a lot of cyber crimes events happened as well. The number of security events is too high for only human can manage, we need to train machines to be able to detect cyber threats. To build the machine that can understand threats, we need the standard to store the cybersecurity information, we need threats' information extraction system. We proposed to invent the cybersecurity event ontology as the medium in storing and sharing cybersecurity event knowledge. Also, we will develop event extraction system that can extract cyber threats events from text. We will extend our previous work on event extraction that detected human activity events from news and discussion forums. A new set of features and learning algorithms will be introduced to improve the performance and adapt the system to cybersecurity domain. We believe that this dissertation will be useful for cybersecurity management in the future. It will quickly grasp cybersecurity events out of text and fill in the event ontology. So we can compete with tons of cybersecurity events that happen in everyday.


\par\vfil

